%! Author = Tim Häberlein
%! Organisation = Technische Universität Dresden, Professur Fahrzeugmechatronik
%! Date = 20.03.2024


% Preamble
\documentclass[class=tudscrartcl, crop=false, cdfont=false, cd=true]{standalone}

% Packages
\usepackage{../packages}

\usepackage{xspace}
\usepackage{xcolor}

\usepackage{pgfplots}
\usepackage{pgf}
\pgfplotsset{compat=1.18}

%\setpythontexworkingdir{../src/mwe_matplotlib/pythontex}
\setpythontexcontext{textwidth=\the\textwidth, textheight=\the\textheight}

\def\mathdefault#1{#1}
\everymath=\expandafter{\the\everymath\displaystyle}

\makeatletter\@ifpackageloaded{underscore}{}{\usepackage[strings]{underscore}}\makeatother
% logical markups
\newcommand*{\python}{\mbox{\textsc{Python}}\xspace}
\newcommand*{\matplotlib}{\mbox{\texttt{Matplotlib}}\xspace}
\newcommand*{\pythontex}{\mbox{\textsc{Python}}\TeX\xspace}

% listings styles
\lstdefinestyle{latexstyle}{
    basicstyle=\ttfamily\footnotesize,
    numbers=left,
    numberstyle=\tiny\color{gray},
    stepnumber=1,
    numbersep=1em,
    tabsize=1,
    extendedchars=true,
    breaklines=true,
    keywordstyle=\color{cdblue},
    stringstyle=\color{red},
    identifierstyle=\color{black},
    commentstyle=\color{cdgreen},
    showspaces=false,
    showstringspaces=false,
    captionpos=b,
    frame=lines,
    rulecolor=\color{cdgrey},
    xleftmargin=4em,
    xrightmargin=2em,
    framexleftmargin=2em,
    aboveskip=1em,  % Abstand vor dem Codeblock
    belowskip=1em,  % Abstand nach dem Codeblock
    morekeywords={begin, usepackage, documentclass}
}

\lstdefinestyle{pythonstyle}{
    language=Python,
    backgroundcolor=\color{cdgrey!10},
    commentstyle=\color{cddarkgreen},
    keywordstyle=\color{cdpurple},
    numberstyle=\tiny\color{cdgrey},
    stringstyle=\color{cdgreen},
    basicstyle=\ttfamily\footnotesize,
    breakatwhitespace=false,
    breaklines=true,
    captionpos=b,
    keepspaces=true,
    numbers=left,
    numbersep=1em,
    showspaces=false,
    showstringspaces=false,
    showtabs=false,
    tabsize=1,
    xleftmargin=4em,                     % Einrückung von links
%frame=l,                            % Rahmen auf der linken Seite des Listings
%framesep=5em,                       % Abstand zwischen Rahmen und Text
    framexleftmargin=2em,             % Abstand des Rahmens zur tatsächlichen Einrückung
    xrightmargin=2em,
    aboveskip=1em,  % Abstand vor dem Codeblock
    belowskip=1em,  % Abstand nach dem Codeblock
}

\lstdefinestyle{bashstyle}{
    backgroundcolor=\color{cdgrey},
    commentstyle=\color{cdgreen},
    keywordstyle=\color{cdgrey!10}\textbf,
    numberstyle=\tiny\color{white},
    stringstyle=\color{cdgreen},
    basicstyle=\ttfamily\footnotesize\color{cdgrey!20},
    breakatwhitespace=false,
    breaklines=true,
    captionpos=b,
    keepspaces=true,
    numbers=left,
    numbersep=1em,
    showspaces=false,
    showstringspaces=false,
    showtabs=false,
    tabsize=1,
    xleftmargin=4em,                     % Einrückung von links
%frame=l,                            % Rahmen auf der linken Seite des Listings
%framesep=5em,                       % Abstand zwischen Rahmen und Text
    framexleftmargin=2em,             % Abstand des Rahmens zur tatsächlichen Einrückung
    xrightmargin=2em,
    aboveskip=1em,  % Abstand vor dem Codeblock
    belowskip=1em,  % Abstand nach dem Codeblock
    morecomment=[l]{::},              % Definiert den Kommentarstil für ::
    morekeywords={python, pip, pythontex},              % Definiert den Kommentarstil für
}
\lstset{
    inputencoding=utf8,
    extendedchars=true,
    literate={ä}{{\"a}}1
        {ö}{{\"o}}1
        {ü}{{\"u}}1
        {Ä}{{\"A}}1
        {Ö}{{\"O}}1
        {Ü}{{\"U}}1
        {ß}{{\ss}}1
}

% Document
\begin{document}

    \section{Import von PGF-Plots über \matplotlib mittels \pythontex}\label{sec:import-von-pgf-plots-uber-matplotlib-mittels-pythontex}
    \pythontex ist ein \LaTeX-Paket, welches es ermöglicht, \python-Code in \LaTeX-Dokumente einzubinden und auszuführen.
    Dies ermöglicht den direkten Import des in \autoref{lst:plot-code} aus \autoref{sec:vergleich-von-pgf-und-pdf} erzeugten Plots.

    \subsection{Vorbereitungen in \LaTeX für \pythontex}\label{subsec:vorbereitungen-in-latex-fuer-pythontex}
    \pythontex wird mit mittels des folgendem Codes importiert:
\begin{lstlisting}[language=TeX, style=latexstyle, caption=Einbinden von \pythontex,label={lst:pythontex-packages}]
\usepackage{pythontex}
\end{lstlisting}

    \subsection{Vorbereitungen in Windows und \python für \pythontex}\label{subsec:vorbereitungen-in-python-fuer-pythontex}

    Für die Verwendung von \python empfiehlt sich immer die Verwendung eines virtuellen Environments (venv) für jedes spezifische Projekt, da so Paket-Abhängigkeiten sauber gehalten werden können und das main-\python-Environment nicht geändert wird.
    Die Erstellung eines venv erfolgt unter Windows mittels folgendem Befehl: \lstinline[language=bash]|python -m venv /path/to/venv|.

    Im nächsten Schritt müssen relevante Pakete in dieses venv installiert werden.
    Dazu wird das venv aktiviert und anschließend mittels des Paketmanagers pip die Pakete in dieses venv installiert.

    Die \python-Umgebung mit \pythontex benötigt im vorliegenden Fall die folgenden Pakete:
    \begin{description}
        \item[\texttt{pygments}] für die Syntax-Highlighting und
        \item[\texttt{matplotlib}] für die Erstellung von Plots.
    \end{description}

    Die Installation und Einrichtung kann mittels des folgenden Codes erfolgen:
\begin{lstlisting}[language=bash, style=bashstyle, caption=Installation von \texttt{pygments} für \pythontex,label={lst:pygments-installation}]
:: installation python venv
python -m venv C:\Users\example_user\IdeaProjects\tex_test\venv
:: activate venv
C:\Users\example_user\IdeaProjects\tex_test\venv\Scripts\activate.bat
:: install python-packages
pip install pygments
pip install matplotlib
\end{lstlisting}

    Der Aufruf des \pythontex Compilers ist in \autoref{lst:pythontex-compiler} im Allgemeinen dargestellt.
    \begin{lstlisting}[language=bash, style=bashstyle, caption={Aufruf des \pythontex Compilers}, label={lst:pythontex-compiler}]
:: Aufruf des pythontex-Compilers im allgemeinen
pythontex --interpreter python:<path/to/venv> <tex_filename>
\end{lstlisting}
    Zu Begin muss jedoch auch das venv aktiviert werden, um den \pythontex-Compiler korrekt ausführen zu können.
    Sollte zusätzlich noch ein output-directory mittels des pdflatex-Compilers angegeben sein, so muss auch der \pythontex-Compiler diesen kennen.
    In diesem Fall empfiehlt sich die Verwendung eines kleinen Scriptes mittels eines Argumentes (\#1) für das \LaTeX-File, welches dann den \pythontex-Compiler aufruft.
    Für den vorliegenden Fall könnte dieses Script wie folgt aussehen:
\begin{lstlisting}[language=bash, style=bashstyle, caption=Script für den Aufruf des \pythontex-Compilers,label={lst:pythontex-start-script}]
:: Wechsel in das output-directory des pdflatex-Compilers
cd C:\Users\s7285521\IdeaProjects\_Diss\tex_test\out
:: Aktivierung des venv und Aufruf des pythontex-Compilers
..\venv\Scripts\activate.bat && pythontex --interpreter python:C:\\Users\\s7285521\\IdeaProjects\\_Diss\\tex_test\\venv\\Scripts\\python.exe %1
\end{lstlisting}

    Der typische Kompilierungsvorgang sieht dann wie folgt aus: pdflatex - pythontex - pdflatex.
    Sollte sich im \python-Code nicht geändert haben, genügt danach ein pdflatex Durchgang.

    \subsection{Einbindung und Verwendung von \pythontex in \LaTeX}\label{subsec:einbindung-und-verwendung-von-pythontex-in-latex}
    In der Regel erfolgt die Einbindung von \pythontex in \LaTeX{} mittels der \texttt{pycode}-Umgebung.
    Die Verwendung einer Konsolen-Umgebung wird mittels \texttt{pyconsole} realisiert.

    Die Ausgabe des \autoref{lst:pythontex-verwendung} ist darunter dargestellt.
    Weder die Konsolenumgebung noch die Inline-Konsolen-Umgebung brechen Text auf eine neue Zeile um.

    \begin{lstlisting}[language=TeX, style=latexstyle,label={lst:pythontex-verwendung}, caption={\pythontex-Verwendung}]
\documentclass[class=tudscrartcl, cdfont=false]{standalone}
\usepackage{pythontex}
\begin{document}
    \python-Code kann mittels \lstinline[language=Tex, style=latexstyle]|\begin{pyconsole}| und \lstinline[language=Tex, style=latexstyle]|\end{pyconsole}|  in \LaTeX{} eingebunden und ausgefuehrt werden.
    Dabei sowohl Ein- als auch Ausgaben dargestellt, als ob es sich um eine richtige Konsole handeln wuerde.
\begin{pyconsole}
import os
os.getcwd()
\end{pyconsole}
    Der \python-Code innerhalb der \texttt{pyconsole}-Umgebung darf dabei nicht eingerueckt werden.
    Mithilfe des \lstinline[language=Tex, style=latexstyle]|\pycon{}|-Befehls kann der Output auch inline dargestellt werden.
    Inline dargestellt ist das \python-Workingdirectory: \\ \pycon{os.getcwd()}
\end{document}
\end{lstlisting}

    \python-Code kann mittels \lstinline[language=Tex, style=latexstyle]|\begin{pyconsole}| und \lstinline[language=Tex, style=latexstyle]|\end{pyconsole}|  in  \LaTeX{} eingebunden und ausgeführt werden.
    Dabei werden sowohl Ein- als auch Ausgaben dargestellt, als ob es sich um eine richtige Konsole handeln würde.
\begin{pyconsole}
import os
os.getcwd()
\end{pyconsole}
    Der \python-Code innerhalb der \texttt{pyconsole}-Umgebung darf dabei nicht eingerückt werden.
    Mithilfe des \lstinline[language=Tex, style=latexstyle]|\pycon{}|-Befehls kann der Output auch inline dargestellt werden.
    Inline dargestellt ist das \python-Workingdirectory: \\ \pycon{os.getcwd()}

    Bei der Verwendung von Variablen ist darauf zu achten, dass diese nur in der jeweiligen Umgebung definiert sind und verwendet werden können.
    Definieren wir noch eine Variable \texttt{a} in einer \texttt{pycode}-Umgebung, so ist diese nur innerhalb dieser Umgebung definiert und kann nicht in einer \texttt{pyconsole}-Umgebung verwendet werden.

\begin{lstlisting}[language=TeX, style=latexstyle, caption={\pythontex-Verwendung von Variablen}, label={lst:pythontex-variables}]
\begin{pycode}
a = 3
\end{pycode}
\begin{pyconsole}
a = 2
a
\end{pyconsole}
\end{lstlisting}

    \begin{pycode}
a = 3
    \end{pycode}
    \begin{pyconsole}
a = 2
a
    \end{pyconsole}

    Nach dem Aufruf des Codes aus \autoref{lst:pythontex-variables} wird die \texttt{pyconsole}-Umgebung ausgegeben. Da es in der \texttt{pycode}-Umgebung keine Ausgabe (z.\@\,B.\@ einen print-Befehl) gibt, wird diese nicht dargestellt.
    Beide Variablen können wir jetzt in der jeweiligen in-Line-Umgenbung verwenden:
    \begin{itemize}
        \item \lstinline[language=TeX, style=latexstyle]|\pycon{a*2}|: \pycon{a*2}
        \item \lstinline[language=TeX, style=latexstyle]|\py{a*2}|: \py{a*2}
    \end{itemize}

    \subsection{Übergabe von Variablen zwischen \LaTeX{} und \python}\label{subsec:uebergabe-von-variablen-zwischen-latex-und-python}
    Die Übergabe von Variablen zwischen \LaTeX{} und \python erfolgt mittels:
    \begin{lstlisting}[language=TeX, style=latexstyle, caption={Übergabe von Variablen zwischen \LaTeX{} und \python}, label={lst:pythontex-variable-passing}]
\documentclass[class=tudscrartcl, cdfont=false]{standalone}
\usepackage{pythontex}
\setpythontexcontext{textwidth=\the\textwidth, textheight=\the\textheight}
    \begin{document}
        \py{pytex.context['textwidth']} und \py{pytex.context.textheight} sind dabei die Variablen, die von \LaTeX{} an \python uebergeben werden.
        Und können in mittels vorhandener Funktionen in Inches umgerechnet werden: \py{pytex.pt_to_in(pytex.context['textwidth'])}.
        Der Rückgabetyp ist dann \py{type(pytex.pt_to_in(pytex.context['textwidth']))}.
    \end{document}
\end{lstlisting}

    \py{pytex.context['textwidth']} und \py{pytex.context.textheight} sind dabei die Variablen, die von \LaTeX{} an \python übergeben werden.
    Und können in mittels vorhandener Funktionen in Inches umgerechnet werden: \py{pytex.pt_to_in(pytex.context['textwidth'])}.
    Der Rückgabetyp ist dann \py{type(pytex.pt_to_in(pytex.context['textwidth']))}.

    Dabei muss der Aufruf von \lstinline[language=TeX, style=latexstyle]|\setpythontexcontext{}| in der Präambel erfolgen.
    Die Variablen stehen dann ich allen \texttt{pycode}-Umgebungen zur Verfügung.

    \subsection{Sessions}\label{subsec:sessions}
    Sofern der \texttt{pycode}-Umgebung kein optionales Argument als Session-Name übergeben wird (s. \autoref{lst:pythontex-sessions}), laufen alle Umgebungen nacheinander ab und können auf die gleichen Variablen zugreifen.
    Soll spezifischer Code für alle Sessions ausgeführt werden, so kann dies in der \texttt{pythontexcustomcode}-Umgebung erfolgen (s. \autoref{lst:pythontex-custom-code-env}).

    \begin{lstlisting}[language=TeX, style=latexstyle, caption={\pythontex-Sessions},label={lst:pythontex-sessions}]
\begin{pycode}[<session-name>]
    <code>
\end{pycode}
\end{lstlisting}

    \begin{lstlisting}[language=TeX, style=latexstyle, caption={\pythontex-Costum Code Environment},label={lst:pythontex-custom-code-env}]
\begin{pythontexcustomcode}{py}
    <code>
\end{pythontexcustomcode}
\end{lstlisting}

    \subsection{Working-Directory}\label{subsec:working-directory}
    Bei der Verwendung von \pythontex bietet es sich an, dass Working-Directory von \python in das Verzeichnis zu setzen, in dem sich die \LaTeX{}-Datei befindet und nicht im output-directory des pdflatex-Compilers zu arbeiten.
    So erfolgen imports von diesem Verzeichnis und outputs werden auch dort gespeichert.
    Gerade bei der Erstellung von Plot mittels \matplotlib ist dies von Vorteil, da der Output direkt im Verzeichnis der \LaTeX-Datei liegt und nicht erst in das output-directory gewechselt werden muss.
    Der Wechsel kann dabei innerhalb der \texttt{pycode}-Umgebung  mittels \python oder durch den \LaTeX{}-Befehl \lstinline[language=TeX, style=latexstyle]|\setpythontexworkingdir{<path>}| erfolgen.
    Ein Beispiel ist in \autoref{lst:pythontex-set-workingdir} dargestellt.

\begin{lstlisting}[language=TeX, style=latexstyle, caption={Setzen des Working-Directory für \pythontex},label={lst:pythontex-set-workingdir}]
\documentclass[class=tudscrartcl, cdfont=false]{standalone}
\usepackage{pythontex}
\setpythontexworkingdir{../src/mwe_matplotlib/pythontex}
\begin{document}
    \begin{pyconsole}
    import os
    os.getcwd()
    \end{pyconsole}
\end{document}
\end{lstlisting}

    Da diese Einstellung jedoch nur in der Präambel erfolgen kann, bietet es sich diese Option für komplexere Projekte mit mehreren \texttt{.tex}-Dateien nicht an.
    Der \python-Compiler wird nur einmal aufgerufen und hat dementsprechend dann für alle  Files das gleiche \python-Working-directory.
    In diesem Fall bietet es sich an, das Working-Directory innerhalb der \texttt{pycode}-Umgebung zu setzen oder alle Dateien im out-directory von pdflatex zu halten.
    Für den Fall, dass das \python-Working-directory innerhalt der \texttt{pycode}-Umgebung gesetzt werden soll, ist dies in \autoref{lst:pythontex-workingdir} dargestellt.

\begin{lstlisting}[language=TeX, style=latexstyle, caption={Setzen des Working-Directory für \pythontex durch \python},label={lst:pythontex-workingdir}]
\documentclass[class=tudscrartcl, cdfont=false]{standalone}
\usepackage{pythontex}
\begin{document}
    \begin{pyconsole}
import os
os.getcwd()
os.chdir('../src/mwe_matplotlib/pythontex')
os.getcwd()
    \end{pyconsole}
\end{document}
\end{lstlisting}

    Die Ausgabe sieht dann wie folgt aus:
    \begin{pyconsole}
import os
os.getcwd()
os.chdir('../src/mwe_matplotlib/pythontex')
os.getcwd()
    \end{pyconsole}

    \subsection{Einbindung in \LaTeX{}\label{subsec:pythontex-einbindung-in-latex}}
    Ein Beispiel-Plot mittels \matplotlib und \pythontex ist in \autoref{lst:pythontex-pgf-plot} dargestellt.

    \subsection{Verweise}\label{subsec:verweise-pythontex}
    Die Informationen wurden folgenden Quellen entnommen:
    \begin{itemize}
        \item \href{https://iopscience.iop.org/article/10.1088/1749-4699/8/1/014010}{PythonTeX: reproducible documents with LaTeX, Python, and more}
        Paper des Autors (Geoffrey M Poore) von 2015 in Computational Science \& Discovery, Volume 8, Number 1, DOI: 10.1088/1749-4699/8/1/014010.
        Übersicht zur Verwendung von \pythontex.

        \item \href{https://github.com/gpoore/pythontex}{GitHub \pythontex}
        offizielle GitHub-Seite von \pythontex.

        \item \href{https://ctan.org/pkg/pythontex}{CTAN \pythontex}
        offizielle Paketdokumentation.

        \item \href{https://punkt-akademie.de/wp-content/uploads/2018/10/PythonTeX.pdf}{Dokumentenautomation mit PythonTEX}
        Vorstellung und Beispiele zu \pythontex von Karsten Brodmann, 2018.
    \end{itemize}

    \section{Vergleich PGF und \pythontex Plots}\label{sec:vergleich-pgf-und-pythontex-plots}
    In diesem Abschnitt wird der Vergleich von PGF-Plots und \pythontex-Plots dargestellt.
    Im ersten Schritt werden zuvor jedoch noch 2 verschiedene Möglichkeiten vorgestellt, wie ein Integration in \LaTeX{} erfolgen kann.

    \subsection{Einbindung über \pythontex und \LaTeX}\label{subsec:einbindung-ueber-pythontex-und-latex}

    Die Einbindung des PGF-Plots mittels \pythontex ist in dargestellt \autoref{lst:pythontex-pgf-plot}.
    Dabei wird der Plot mittels \matplotlib durch \pythontex direkt zur Laufzeit erstellt und als PGF-Datei gespeichert.
    Der \python-Code für die Plot Funktion ist in \texttt{PlotBase.py} enthalten.
    Der Inhalt der Datei ist \cref{lst:PlotBase} zu entnehmen.
    Anschließend wird diese Datei in das \LaTeX-Dokument eingebunden.
    Das Ergebnis ist in \cref{fig:pythontex-pgf-plot-einer-sinus-funktion} dargestellt.

    \begin{lstlisting}[language=TeX, style=latexstyle, caption={Einbindung von PGF-Plots mittels \pythontex}, label={lst:pythontex-pgf-plot}]
\begin{pycode}[plot1]
import os, sys
os.chdir('../src/mwe_matplotlib/pythontex')
sys.path.append(os.getcwd())

from PlotBase import plot_sinus
filename = 'sinus_07.pgf'
plot_sinus(0.7*pytex.pt_to_in(pytex.context['textwidth']), filename)
#pytex.add_created(filename)
\end{pycode}
\begin{figure}[htb!]
    \centering
    %\setlength{\figurewidth}{0.8\columnwidth}% Skalierung der Breite
    \subimport{}{sinus_07.pgf}
    \caption{\pythontex PGF-Plot einer Sinus-Funktion.}
\end{figure}
\end{lstlisting}

    \begin{pycode}
import os, sys
os.chdir('../src/mwe_matplotlib/pythontex')
sys.path.append(os.getcwd())

from PlotBase import plot_sinus
filename = 'sinus_07.pgf'
plot_sinus(0.7*pytex.pt_to_in(pytex.context['textwidth']), filename)
#pytex.add_created(filename)
    \end{pycode}

    \begin{figure}[htb!]
        \centering
        %\setlength{\figurewidth}{0.8\columnwidth}% Skalierung der Breite
        %\includegraphics{sinus_09.pdf}
        %% Creator: Matplotlib, PGF backend
%%
%% To include the figure in your LaTeX document, write
%%   \input{<filename>.pgf}
%%
%% Make sure the required packages are loaded in your preamble
%%   \usepackage{pgf}
%%
%% Also ensure that all the required font packages are loaded; for instance,
%% the lmodern package is sometimes necessary when using math font.
%%   \usepackage{lmodern}
%%
%% Figures using additional raster images can only be included by \input if
%% they are in the same directory as the main LaTeX file. For loading figures
%% from other directories you can use the `import` package
%%   \usepackage{import}
%%
%% and then include the figures with
%%   \import{<path to file>}{<filename>.pgf}
%%
%% Matplotlib used the following preamble
%%   \def\mathdefault#1{#1}
%%   \everymath=\expandafter{\the\everymath\displaystyle}
%%   
%%   \makeatletter\@ifpackageloaded{underscore}{}{\usepackage[strings]{underscore}}\makeatother
%%
\begingroup%
\makeatletter%
\begin{pgfpicture}%
\pgfpathrectangle{\pgfpointorigin}{\pgfqpoint{4.309450in}{2.625040in}}%
\pgfusepath{use as bounding box, clip}%
\begin{pgfscope}%
\pgfsetbuttcap%
\pgfsetmiterjoin%
\definecolor{currentfill}{rgb}{1.000000,1.000000,1.000000}%
\pgfsetfillcolor{currentfill}%
\pgfsetlinewidth{0.000000pt}%
\definecolor{currentstroke}{rgb}{1.000000,1.000000,1.000000}%
\pgfsetstrokecolor{currentstroke}%
\pgfsetdash{}{0pt}%
\pgfpathmoveto{\pgfqpoint{0.000000in}{0.000000in}}%
\pgfpathlineto{\pgfqpoint{4.309450in}{0.000000in}}%
\pgfpathlineto{\pgfqpoint{4.309450in}{2.625040in}}%
\pgfpathlineto{\pgfqpoint{0.000000in}{2.625040in}}%
\pgfpathlineto{\pgfqpoint{0.000000in}{0.000000in}}%
\pgfpathclose%
\pgfusepath{fill}%
\end{pgfscope}%
\begin{pgfscope}%
\pgfsetbuttcap%
\pgfsetmiterjoin%
\definecolor{currentfill}{rgb}{1.000000,1.000000,1.000000}%
\pgfsetfillcolor{currentfill}%
\pgfsetlinewidth{0.000000pt}%
\definecolor{currentstroke}{rgb}{0.000000,0.000000,0.000000}%
\pgfsetstrokecolor{currentstroke}%
\pgfsetstrokeopacity{0.000000}%
\pgfsetdash{}{0pt}%
\pgfpathmoveto{\pgfqpoint{0.100000in}{0.100000in}}%
\pgfpathlineto{\pgfqpoint{4.140005in}{0.100000in}}%
\pgfpathlineto{\pgfqpoint{4.140005in}{2.131522in}}%
\pgfpathlineto{\pgfqpoint{0.100000in}{2.131522in}}%
\pgfpathlineto{\pgfqpoint{0.100000in}{0.100000in}}%
\pgfpathclose%
\pgfusepath{fill}%
\end{pgfscope}%
\begin{pgfscope}%
\pgfpathrectangle{\pgfqpoint{0.100000in}{0.100000in}}{\pgfqpoint{4.040005in}{2.031522in}}%
\pgfusepath{clip}%
\pgfsetrectcap%
\pgfsetroundjoin%
\pgfsetlinewidth{0.803000pt}%
\definecolor{currentstroke}{rgb}{0.690196,0.690196,0.690196}%
\pgfsetstrokecolor{currentstroke}%
\pgfsetdash{}{0pt}%
\pgfpathmoveto{\pgfqpoint{0.283637in}{0.100000in}}%
\pgfpathlineto{\pgfqpoint{0.283637in}{2.131522in}}%
\pgfusepath{stroke}%
\end{pgfscope}%
\begin{pgfscope}%
\pgfsetbuttcap%
\pgfsetroundjoin%
\definecolor{currentfill}{rgb}{0.447059,0.470588,0.474510}%
\pgfsetfillcolor{currentfill}%
\pgfsetlinewidth{0.803000pt}%
\definecolor{currentstroke}{rgb}{0.447059,0.470588,0.474510}%
\pgfsetstrokecolor{currentstroke}%
\pgfsetdash{}{0pt}%
\pgfsys@defobject{currentmarker}{\pgfqpoint{0.000000in}{-0.048611in}}{\pgfqpoint{0.000000in}{0.000000in}}{%
\pgfpathmoveto{\pgfqpoint{0.000000in}{0.000000in}}%
\pgfpathlineto{\pgfqpoint{0.000000in}{-0.048611in}}%
\pgfusepath{stroke,fill}%
}%
\begin{pgfscope}%
\pgfsys@transformshift{0.283637in}{1.115761in}%
\pgfsys@useobject{currentmarker}{}%
\end{pgfscope}%
\end{pgfscope}%
\begin{pgfscope}%
\definecolor{textcolor}{rgb}{0.447059,0.470588,0.474510}%
\pgfsetstrokecolor{textcolor}%
\pgfsetfillcolor{textcolor}%
\pgftext[x=0.283637in,y=1.018539in,,top]{\color{textcolor}{\rmfamily\fontsize{10.000000}{12.000000}\selectfont\catcode`\^=\active\def^{\ifmmode\sp\else\^{}\fi}\catcode`\%=\active\def%{\%}$-2\pi$}}%
\end{pgfscope}%
\begin{pgfscope}%
\pgfpathrectangle{\pgfqpoint{0.100000in}{0.100000in}}{\pgfqpoint{4.040005in}{2.031522in}}%
\pgfusepath{clip}%
\pgfsetrectcap%
\pgfsetroundjoin%
\pgfsetlinewidth{0.803000pt}%
\definecolor{currentstroke}{rgb}{0.690196,0.690196,0.690196}%
\pgfsetstrokecolor{currentstroke}%
\pgfsetdash{}{0pt}%
\pgfpathmoveto{\pgfqpoint{1.201820in}{0.100000in}}%
\pgfpathlineto{\pgfqpoint{1.201820in}{2.131522in}}%
\pgfusepath{stroke}%
\end{pgfscope}%
\begin{pgfscope}%
\pgfsetbuttcap%
\pgfsetroundjoin%
\definecolor{currentfill}{rgb}{0.447059,0.470588,0.474510}%
\pgfsetfillcolor{currentfill}%
\pgfsetlinewidth{0.803000pt}%
\definecolor{currentstroke}{rgb}{0.447059,0.470588,0.474510}%
\pgfsetstrokecolor{currentstroke}%
\pgfsetdash{}{0pt}%
\pgfsys@defobject{currentmarker}{\pgfqpoint{0.000000in}{-0.048611in}}{\pgfqpoint{0.000000in}{0.000000in}}{%
\pgfpathmoveto{\pgfqpoint{0.000000in}{0.000000in}}%
\pgfpathlineto{\pgfqpoint{0.000000in}{-0.048611in}}%
\pgfusepath{stroke,fill}%
}%
\begin{pgfscope}%
\pgfsys@transformshift{1.201820in}{1.115761in}%
\pgfsys@useobject{currentmarker}{}%
\end{pgfscope}%
\end{pgfscope}%
\begin{pgfscope}%
\definecolor{textcolor}{rgb}{0.447059,0.470588,0.474510}%
\pgfsetstrokecolor{textcolor}%
\pgfsetfillcolor{textcolor}%
\pgftext[x=1.201820in,y=1.018539in,,top]{\color{textcolor}{\rmfamily\fontsize{10.000000}{12.000000}\selectfont\catcode`\^=\active\def^{\ifmmode\sp\else\^{}\fi}\catcode`\%=\active\def%{\%}-$\pi$}}%
\end{pgfscope}%
\begin{pgfscope}%
\pgfpathrectangle{\pgfqpoint{0.100000in}{0.100000in}}{\pgfqpoint{4.040005in}{2.031522in}}%
\pgfusepath{clip}%
\pgfsetrectcap%
\pgfsetroundjoin%
\pgfsetlinewidth{0.803000pt}%
\definecolor{currentstroke}{rgb}{0.690196,0.690196,0.690196}%
\pgfsetstrokecolor{currentstroke}%
\pgfsetdash{}{0pt}%
\pgfpathmoveto{\pgfqpoint{2.120003in}{0.100000in}}%
\pgfpathlineto{\pgfqpoint{2.120003in}{2.131522in}}%
\pgfusepath{stroke}%
\end{pgfscope}%
\begin{pgfscope}%
\pgfsetbuttcap%
\pgfsetroundjoin%
\definecolor{currentfill}{rgb}{0.447059,0.470588,0.474510}%
\pgfsetfillcolor{currentfill}%
\pgfsetlinewidth{0.803000pt}%
\definecolor{currentstroke}{rgb}{0.447059,0.470588,0.474510}%
\pgfsetstrokecolor{currentstroke}%
\pgfsetdash{}{0pt}%
\pgfsys@defobject{currentmarker}{\pgfqpoint{0.000000in}{-0.048611in}}{\pgfqpoint{0.000000in}{0.000000in}}{%
\pgfpathmoveto{\pgfqpoint{0.000000in}{0.000000in}}%
\pgfpathlineto{\pgfqpoint{0.000000in}{-0.048611in}}%
\pgfusepath{stroke,fill}%
}%
\begin{pgfscope}%
\pgfsys@transformshift{2.120003in}{1.115761in}%
\pgfsys@useobject{currentmarker}{}%
\end{pgfscope}%
\end{pgfscope}%
\begin{pgfscope}%
\definecolor{textcolor}{rgb}{0.447059,0.470588,0.474510}%
\pgfsetstrokecolor{textcolor}%
\pgfsetfillcolor{textcolor}%
\pgftext[x=2.120003in,y=1.018539in,,top]{\color{textcolor}{\rmfamily\fontsize{10.000000}{12.000000}\selectfont\catcode`\^=\active\def^{\ifmmode\sp\else\^{}\fi}\catcode`\%=\active\def%{\%}0}}%
\end{pgfscope}%
\begin{pgfscope}%
\pgfpathrectangle{\pgfqpoint{0.100000in}{0.100000in}}{\pgfqpoint{4.040005in}{2.031522in}}%
\pgfusepath{clip}%
\pgfsetrectcap%
\pgfsetroundjoin%
\pgfsetlinewidth{0.803000pt}%
\definecolor{currentstroke}{rgb}{0.690196,0.690196,0.690196}%
\pgfsetstrokecolor{currentstroke}%
\pgfsetdash{}{0pt}%
\pgfpathmoveto{\pgfqpoint{3.038186in}{0.100000in}}%
\pgfpathlineto{\pgfqpoint{3.038186in}{2.131522in}}%
\pgfusepath{stroke}%
\end{pgfscope}%
\begin{pgfscope}%
\pgfsetbuttcap%
\pgfsetroundjoin%
\definecolor{currentfill}{rgb}{0.447059,0.470588,0.474510}%
\pgfsetfillcolor{currentfill}%
\pgfsetlinewidth{0.803000pt}%
\definecolor{currentstroke}{rgb}{0.447059,0.470588,0.474510}%
\pgfsetstrokecolor{currentstroke}%
\pgfsetdash{}{0pt}%
\pgfsys@defobject{currentmarker}{\pgfqpoint{0.000000in}{-0.048611in}}{\pgfqpoint{0.000000in}{0.000000in}}{%
\pgfpathmoveto{\pgfqpoint{0.000000in}{0.000000in}}%
\pgfpathlineto{\pgfqpoint{0.000000in}{-0.048611in}}%
\pgfusepath{stroke,fill}%
}%
\begin{pgfscope}%
\pgfsys@transformshift{3.038186in}{1.115761in}%
\pgfsys@useobject{currentmarker}{}%
\end{pgfscope}%
\end{pgfscope}%
\begin{pgfscope}%
\definecolor{textcolor}{rgb}{0.447059,0.470588,0.474510}%
\pgfsetstrokecolor{textcolor}%
\pgfsetfillcolor{textcolor}%
\pgftext[x=3.038186in,y=1.018539in,,top]{\color{textcolor}{\rmfamily\fontsize{10.000000}{12.000000}\selectfont\catcode`\^=\active\def^{\ifmmode\sp\else\^{}\fi}\catcode`\%=\active\def%{\%}$\pi$}}%
\end{pgfscope}%
\begin{pgfscope}%
\pgfpathrectangle{\pgfqpoint{0.100000in}{0.100000in}}{\pgfqpoint{4.040005in}{2.031522in}}%
\pgfusepath{clip}%
\pgfsetrectcap%
\pgfsetroundjoin%
\pgfsetlinewidth{0.803000pt}%
\definecolor{currentstroke}{rgb}{0.690196,0.690196,0.690196}%
\pgfsetstrokecolor{currentstroke}%
\pgfsetdash{}{0pt}%
\pgfpathmoveto{\pgfqpoint{3.956369in}{0.100000in}}%
\pgfpathlineto{\pgfqpoint{3.956369in}{2.131522in}}%
\pgfusepath{stroke}%
\end{pgfscope}%
\begin{pgfscope}%
\pgfsetbuttcap%
\pgfsetroundjoin%
\definecolor{currentfill}{rgb}{0.447059,0.470588,0.474510}%
\pgfsetfillcolor{currentfill}%
\pgfsetlinewidth{0.803000pt}%
\definecolor{currentstroke}{rgb}{0.447059,0.470588,0.474510}%
\pgfsetstrokecolor{currentstroke}%
\pgfsetdash{}{0pt}%
\pgfsys@defobject{currentmarker}{\pgfqpoint{0.000000in}{-0.048611in}}{\pgfqpoint{0.000000in}{0.000000in}}{%
\pgfpathmoveto{\pgfqpoint{0.000000in}{0.000000in}}%
\pgfpathlineto{\pgfqpoint{0.000000in}{-0.048611in}}%
\pgfusepath{stroke,fill}%
}%
\begin{pgfscope}%
\pgfsys@transformshift{3.956369in}{1.115761in}%
\pgfsys@useobject{currentmarker}{}%
\end{pgfscope}%
\end{pgfscope}%
\begin{pgfscope}%
\definecolor{textcolor}{rgb}{0.447059,0.470588,0.474510}%
\pgfsetstrokecolor{textcolor}%
\pgfsetfillcolor{textcolor}%
\pgftext[x=3.956369in,y=1.018539in,,top]{\color{textcolor}{\rmfamily\fontsize{10.000000}{12.000000}\selectfont\catcode`\^=\active\def^{\ifmmode\sp\else\^{}\fi}\catcode`\%=\active\def%{\%}$2\pi$}}%
\end{pgfscope}%
\begin{pgfscope}%
\pgfpathrectangle{\pgfqpoint{0.100000in}{0.100000in}}{\pgfqpoint{4.040005in}{2.031522in}}%
\pgfusepath{clip}%
\pgfsetrectcap%
\pgfsetroundjoin%
\pgfsetlinewidth{0.803000pt}%
\definecolor{currentstroke}{rgb}{0.690196,0.690196,0.690196}%
\pgfsetstrokecolor{currentstroke}%
\pgfsetdash{}{0pt}%
\pgfpathmoveto{\pgfqpoint{0.100000in}{0.192226in}}%
\pgfpathlineto{\pgfqpoint{4.140005in}{0.192226in}}%
\pgfusepath{stroke}%
\end{pgfscope}%
\begin{pgfscope}%
\pgfsetbuttcap%
\pgfsetroundjoin%
\definecolor{currentfill}{rgb}{0.447059,0.470588,0.474510}%
\pgfsetfillcolor{currentfill}%
\pgfsetlinewidth{0.803000pt}%
\definecolor{currentstroke}{rgb}{0.447059,0.470588,0.474510}%
\pgfsetstrokecolor{currentstroke}%
\pgfsetdash{}{0pt}%
\pgfsys@defobject{currentmarker}{\pgfqpoint{-0.048611in}{0.000000in}}{\pgfqpoint{-0.000000in}{0.000000in}}{%
\pgfpathmoveto{\pgfqpoint{-0.000000in}{0.000000in}}%
\pgfpathlineto{\pgfqpoint{-0.048611in}{0.000000in}}%
\pgfusepath{stroke,fill}%
}%
\begin{pgfscope}%
\pgfsys@transformshift{2.120003in}{0.192226in}%
\pgfsys@useobject{currentmarker}{}%
\end{pgfscope}%
\end{pgfscope}%
\begin{pgfscope}%
\definecolor{textcolor}{rgb}{0.447059,0.470588,0.474510}%
\pgfsetstrokecolor{textcolor}%
\pgfsetfillcolor{textcolor}%
\pgftext[x=1.737286in, y=0.144000in, left, base]{\color{textcolor}{\rmfamily\fontsize{10.000000}{12.000000}\selectfont\catcode`\^=\active\def^{\ifmmode\sp\else\^{}\fi}\catcode`\%=\active\def%{\%}$\mathdefault{\ensuremath{-}1.0}$}}%
\end{pgfscope}%
\begin{pgfscope}%
\pgfpathrectangle{\pgfqpoint{0.100000in}{0.100000in}}{\pgfqpoint{4.040005in}{2.031522in}}%
\pgfusepath{clip}%
\pgfsetrectcap%
\pgfsetroundjoin%
\pgfsetlinewidth{0.803000pt}%
\definecolor{currentstroke}{rgb}{0.690196,0.690196,0.690196}%
\pgfsetstrokecolor{currentstroke}%
\pgfsetdash{}{0pt}%
\pgfpathmoveto{\pgfqpoint{0.100000in}{0.653993in}}%
\pgfpathlineto{\pgfqpoint{4.140005in}{0.653993in}}%
\pgfusepath{stroke}%
\end{pgfscope}%
\begin{pgfscope}%
\pgfsetbuttcap%
\pgfsetroundjoin%
\definecolor{currentfill}{rgb}{0.447059,0.470588,0.474510}%
\pgfsetfillcolor{currentfill}%
\pgfsetlinewidth{0.803000pt}%
\definecolor{currentstroke}{rgb}{0.447059,0.470588,0.474510}%
\pgfsetstrokecolor{currentstroke}%
\pgfsetdash{}{0pt}%
\pgfsys@defobject{currentmarker}{\pgfqpoint{-0.048611in}{0.000000in}}{\pgfqpoint{-0.000000in}{0.000000in}}{%
\pgfpathmoveto{\pgfqpoint{-0.000000in}{0.000000in}}%
\pgfpathlineto{\pgfqpoint{-0.048611in}{0.000000in}}%
\pgfusepath{stroke,fill}%
}%
\begin{pgfscope}%
\pgfsys@transformshift{2.120003in}{0.653993in}%
\pgfsys@useobject{currentmarker}{}%
\end{pgfscope}%
\end{pgfscope}%
\begin{pgfscope}%
\definecolor{textcolor}{rgb}{0.447059,0.470588,0.474510}%
\pgfsetstrokecolor{textcolor}%
\pgfsetfillcolor{textcolor}%
\pgftext[x=1.737286in, y=0.605768in, left, base]{\color{textcolor}{\rmfamily\fontsize{10.000000}{12.000000}\selectfont\catcode`\^=\active\def^{\ifmmode\sp\else\^{}\fi}\catcode`\%=\active\def%{\%}$\mathdefault{\ensuremath{-}0.5}$}}%
\end{pgfscope}%
\begin{pgfscope}%
\pgfpathrectangle{\pgfqpoint{0.100000in}{0.100000in}}{\pgfqpoint{4.040005in}{2.031522in}}%
\pgfusepath{clip}%
\pgfsetrectcap%
\pgfsetroundjoin%
\pgfsetlinewidth{0.803000pt}%
\definecolor{currentstroke}{rgb}{0.690196,0.690196,0.690196}%
\pgfsetstrokecolor{currentstroke}%
\pgfsetdash{}{0pt}%
\pgfpathmoveto{\pgfqpoint{0.100000in}{1.115761in}}%
\pgfpathlineto{\pgfqpoint{4.140005in}{1.115761in}}%
\pgfusepath{stroke}%
\end{pgfscope}%
\begin{pgfscope}%
\pgfsetbuttcap%
\pgfsetroundjoin%
\definecolor{currentfill}{rgb}{0.447059,0.470588,0.474510}%
\pgfsetfillcolor{currentfill}%
\pgfsetlinewidth{0.803000pt}%
\definecolor{currentstroke}{rgb}{0.447059,0.470588,0.474510}%
\pgfsetstrokecolor{currentstroke}%
\pgfsetdash{}{0pt}%
\pgfsys@defobject{currentmarker}{\pgfqpoint{-0.048611in}{0.000000in}}{\pgfqpoint{-0.000000in}{0.000000in}}{%
\pgfpathmoveto{\pgfqpoint{-0.000000in}{0.000000in}}%
\pgfpathlineto{\pgfqpoint{-0.048611in}{0.000000in}}%
\pgfusepath{stroke,fill}%
}%
\begin{pgfscope}%
\pgfsys@transformshift{2.120003in}{1.115761in}%
\pgfsys@useobject{currentmarker}{}%
\end{pgfscope}%
\end{pgfscope}%
\begin{pgfscope}%
\definecolor{textcolor}{rgb}{0.447059,0.470588,0.474510}%
\pgfsetstrokecolor{textcolor}%
\pgfsetfillcolor{textcolor}%
\pgftext[x=1.845311in, y=1.067536in, left, base]{\color{textcolor}{\rmfamily\fontsize{10.000000}{12.000000}\selectfont\catcode`\^=\active\def^{\ifmmode\sp\else\^{}\fi}\catcode`\%=\active\def%{\%}$\mathdefault{0.0}$}}%
\end{pgfscope}%
\begin{pgfscope}%
\pgfpathrectangle{\pgfqpoint{0.100000in}{0.100000in}}{\pgfqpoint{4.040005in}{2.031522in}}%
\pgfusepath{clip}%
\pgfsetrectcap%
\pgfsetroundjoin%
\pgfsetlinewidth{0.803000pt}%
\definecolor{currentstroke}{rgb}{0.690196,0.690196,0.690196}%
\pgfsetstrokecolor{currentstroke}%
\pgfsetdash{}{0pt}%
\pgfpathmoveto{\pgfqpoint{0.100000in}{1.577528in}}%
\pgfpathlineto{\pgfqpoint{4.140005in}{1.577528in}}%
\pgfusepath{stroke}%
\end{pgfscope}%
\begin{pgfscope}%
\pgfsetbuttcap%
\pgfsetroundjoin%
\definecolor{currentfill}{rgb}{0.447059,0.470588,0.474510}%
\pgfsetfillcolor{currentfill}%
\pgfsetlinewidth{0.803000pt}%
\definecolor{currentstroke}{rgb}{0.447059,0.470588,0.474510}%
\pgfsetstrokecolor{currentstroke}%
\pgfsetdash{}{0pt}%
\pgfsys@defobject{currentmarker}{\pgfqpoint{-0.048611in}{0.000000in}}{\pgfqpoint{-0.000000in}{0.000000in}}{%
\pgfpathmoveto{\pgfqpoint{-0.000000in}{0.000000in}}%
\pgfpathlineto{\pgfqpoint{-0.048611in}{0.000000in}}%
\pgfusepath{stroke,fill}%
}%
\begin{pgfscope}%
\pgfsys@transformshift{2.120003in}{1.577528in}%
\pgfsys@useobject{currentmarker}{}%
\end{pgfscope}%
\end{pgfscope}%
\begin{pgfscope}%
\definecolor{textcolor}{rgb}{0.447059,0.470588,0.474510}%
\pgfsetstrokecolor{textcolor}%
\pgfsetfillcolor{textcolor}%
\pgftext[x=1.845311in, y=1.529303in, left, base]{\color{textcolor}{\rmfamily\fontsize{10.000000}{12.000000}\selectfont\catcode`\^=\active\def^{\ifmmode\sp\else\^{}\fi}\catcode`\%=\active\def%{\%}$\mathdefault{0.5}$}}%
\end{pgfscope}%
\begin{pgfscope}%
\pgfpathrectangle{\pgfqpoint{0.100000in}{0.100000in}}{\pgfqpoint{4.040005in}{2.031522in}}%
\pgfusepath{clip}%
\pgfsetrectcap%
\pgfsetroundjoin%
\pgfsetlinewidth{0.803000pt}%
\definecolor{currentstroke}{rgb}{0.690196,0.690196,0.690196}%
\pgfsetstrokecolor{currentstroke}%
\pgfsetdash{}{0pt}%
\pgfpathmoveto{\pgfqpoint{0.100000in}{2.039296in}}%
\pgfpathlineto{\pgfqpoint{4.140005in}{2.039296in}}%
\pgfusepath{stroke}%
\end{pgfscope}%
\begin{pgfscope}%
\pgfsetbuttcap%
\pgfsetroundjoin%
\definecolor{currentfill}{rgb}{0.447059,0.470588,0.474510}%
\pgfsetfillcolor{currentfill}%
\pgfsetlinewidth{0.803000pt}%
\definecolor{currentstroke}{rgb}{0.447059,0.470588,0.474510}%
\pgfsetstrokecolor{currentstroke}%
\pgfsetdash{}{0pt}%
\pgfsys@defobject{currentmarker}{\pgfqpoint{-0.048611in}{0.000000in}}{\pgfqpoint{-0.000000in}{0.000000in}}{%
\pgfpathmoveto{\pgfqpoint{-0.000000in}{0.000000in}}%
\pgfpathlineto{\pgfqpoint{-0.048611in}{0.000000in}}%
\pgfusepath{stroke,fill}%
}%
\begin{pgfscope}%
\pgfsys@transformshift{2.120003in}{2.039296in}%
\pgfsys@useobject{currentmarker}{}%
\end{pgfscope}%
\end{pgfscope}%
\begin{pgfscope}%
\definecolor{textcolor}{rgb}{0.447059,0.470588,0.474510}%
\pgfsetstrokecolor{textcolor}%
\pgfsetfillcolor{textcolor}%
\pgftext[x=1.845311in, y=1.991071in, left, base]{\color{textcolor}{\rmfamily\fontsize{10.000000}{12.000000}\selectfont\catcode`\^=\active\def^{\ifmmode\sp\else\^{}\fi}\catcode`\%=\active\def%{\%}$\mathdefault{1.0}$}}%
\end{pgfscope}%
\begin{pgfscope}%
\pgfpathrectangle{\pgfqpoint{0.100000in}{0.100000in}}{\pgfqpoint{4.040005in}{2.031522in}}%
\pgfusepath{clip}%
\pgfsetrectcap%
\pgfsetroundjoin%
\pgfsetlinewidth{1.505625pt}%
\definecolor{currentstroke}{rgb}{0.000000,0.188235,0.368627}%
\pgfsetstrokecolor{currentstroke}%
\pgfsetdash{}{0pt}%
\pgfpathmoveto{\pgfqpoint{0.283637in}{1.115761in}}%
\pgfpathlineto{\pgfqpoint{0.320735in}{1.232673in}}%
\pgfpathlineto{\pgfqpoint{0.357833in}{1.347705in}}%
\pgfpathlineto{\pgfqpoint{0.394932in}{1.459004in}}%
\pgfpathlineto{\pgfqpoint{0.432030in}{1.564781in}}%
\pgfpathlineto{\pgfqpoint{0.469128in}{1.663332in}}%
\pgfpathlineto{\pgfqpoint{0.506226in}{1.753073in}}%
\pgfpathlineto{\pgfqpoint{0.543325in}{1.832559in}}%
\pgfpathlineto{\pgfqpoint{0.580423in}{1.900512in}}%
\pgfpathlineto{\pgfqpoint{0.617521in}{1.955838in}}%
\pgfpathlineto{\pgfqpoint{0.654620in}{1.997647in}}%
\pgfpathlineto{\pgfqpoint{0.691718in}{2.025265in}}%
\pgfpathlineto{\pgfqpoint{0.728816in}{2.038250in}}%
\pgfpathlineto{\pgfqpoint{0.765915in}{2.036391in}}%
\pgfpathlineto{\pgfqpoint{0.803013in}{2.019719in}}%
\pgfpathlineto{\pgfqpoint{0.840111in}{1.988502in}}%
\pgfpathlineto{\pgfqpoint{0.877209in}{1.943243in}}%
\pgfpathlineto{\pgfqpoint{0.914308in}{1.884668in}}%
\pgfpathlineto{\pgfqpoint{0.951406in}{1.813722in}}%
\pgfpathlineto{\pgfqpoint{0.988504in}{1.731545in}}%
\pgfpathlineto{\pgfqpoint{1.025603in}{1.639461in}}%
\pgfpathlineto{\pgfqpoint{1.062701in}{1.538949in}}%
\pgfpathlineto{\pgfqpoint{1.099799in}{1.431628in}}%
\pgfpathlineto{\pgfqpoint{1.136898in}{1.319225in}}%
\pgfpathlineto{\pgfqpoint{1.173996in}{1.203548in}}%
\pgfpathlineto{\pgfqpoint{1.211094in}{1.086459in}}%
\pgfpathlineto{\pgfqpoint{1.248192in}{0.969841in}}%
\pgfpathlineto{\pgfqpoint{1.285291in}{0.855571in}}%
\pgfpathlineto{\pgfqpoint{1.322389in}{0.745487in}}%
\pgfpathlineto{\pgfqpoint{1.359487in}{0.641362in}}%
\pgfpathlineto{\pgfqpoint{1.396586in}{0.544869in}}%
\pgfpathlineto{\pgfqpoint{1.433684in}{0.457563in}}%
\pgfpathlineto{\pgfqpoint{1.470782in}{0.380847in}}%
\pgfpathlineto{\pgfqpoint{1.507881in}{0.315956in}}%
\pgfpathlineto{\pgfqpoint{1.544979in}{0.263934in}}%
\pgfpathlineto{\pgfqpoint{1.582077in}{0.225619in}}%
\pgfpathlineto{\pgfqpoint{1.619175in}{0.201626in}}%
\pgfpathlineto{\pgfqpoint{1.656274in}{0.192342in}}%
\pgfpathlineto{\pgfqpoint{1.693372in}{0.197916in}}%
\pgfpathlineto{\pgfqpoint{1.730470in}{0.218259in}}%
\pgfpathlineto{\pgfqpoint{1.767569in}{0.253042in}}%
\pgfpathlineto{\pgfqpoint{1.804667in}{0.301708in}}%
\pgfpathlineto{\pgfqpoint{1.841765in}{0.363471in}}%
\pgfpathlineto{\pgfqpoint{1.878864in}{0.437340in}}%
\pgfpathlineto{\pgfqpoint{1.915962in}{0.522124in}}%
\pgfpathlineto{\pgfqpoint{1.953060in}{0.616460in}}%
\pgfpathlineto{\pgfqpoint{1.990158in}{0.718830in}}%
\pgfpathlineto{\pgfqpoint{2.027257in}{0.827587in}}%
\pgfpathlineto{\pgfqpoint{2.064355in}{0.940981in}}%
\pgfpathlineto{\pgfqpoint{2.101453in}{1.057187in}}%
\pgfpathlineto{\pgfqpoint{2.138552in}{1.174335in}}%
\pgfpathlineto{\pgfqpoint{2.175650in}{1.290541in}}%
\pgfpathlineto{\pgfqpoint{2.212748in}{1.403935in}}%
\pgfpathlineto{\pgfqpoint{2.249847in}{1.512692in}}%
\pgfpathlineto{\pgfqpoint{2.286945in}{1.615062in}}%
\pgfpathlineto{\pgfqpoint{2.324043in}{1.709398in}}%
\pgfpathlineto{\pgfqpoint{2.361142in}{1.794182in}}%
\pgfpathlineto{\pgfqpoint{2.398240in}{1.868050in}}%
\pgfpathlineto{\pgfqpoint{2.435338in}{1.929814in}}%
\pgfpathlineto{\pgfqpoint{2.472436in}{1.978479in}}%
\pgfpathlineto{\pgfqpoint{2.509535in}{2.013263in}}%
\pgfpathlineto{\pgfqpoint{2.546633in}{2.033606in}}%
\pgfpathlineto{\pgfqpoint{2.583731in}{2.039180in}}%
\pgfpathlineto{\pgfqpoint{2.620830in}{2.029896in}}%
\pgfpathlineto{\pgfqpoint{2.657928in}{2.005903in}}%
\pgfpathlineto{\pgfqpoint{2.695026in}{1.967587in}}%
\pgfpathlineto{\pgfqpoint{2.732125in}{1.915566in}}%
\pgfpathlineto{\pgfqpoint{2.769223in}{1.850675in}}%
\pgfpathlineto{\pgfqpoint{2.806321in}{1.773959in}}%
\pgfpathlineto{\pgfqpoint{2.843419in}{1.686652in}}%
\pgfpathlineto{\pgfqpoint{2.880518in}{1.590160in}}%
\pgfpathlineto{\pgfqpoint{2.917616in}{1.486034in}}%
\pgfpathlineto{\pgfqpoint{2.954714in}{1.375951in}}%
\pgfpathlineto{\pgfqpoint{2.991813in}{1.261681in}}%
\pgfpathlineto{\pgfqpoint{3.028911in}{1.145063in}}%
\pgfpathlineto{\pgfqpoint{3.066009in}{1.027973in}}%
\pgfpathlineto{\pgfqpoint{3.103108in}{0.912296in}}%
\pgfpathlineto{\pgfqpoint{3.140206in}{0.799893in}}%
\pgfpathlineto{\pgfqpoint{3.177304in}{0.692573in}}%
\pgfpathlineto{\pgfqpoint{3.214402in}{0.592061in}}%
\pgfpathlineto{\pgfqpoint{3.251501in}{0.499976in}}%
\pgfpathlineto{\pgfqpoint{3.288599in}{0.417800in}}%
\pgfpathlineto{\pgfqpoint{3.325697in}{0.346853in}}%
\pgfpathlineto{\pgfqpoint{3.362796in}{0.288279in}}%
\pgfpathlineto{\pgfqpoint{3.399894in}{0.243019in}}%
\pgfpathlineto{\pgfqpoint{3.436992in}{0.211802in}}%
\pgfpathlineto{\pgfqpoint{3.474091in}{0.195130in}}%
\pgfpathlineto{\pgfqpoint{3.511189in}{0.193272in}}%
\pgfpathlineto{\pgfqpoint{3.548287in}{0.206256in}}%
\pgfpathlineto{\pgfqpoint{3.585385in}{0.233875in}}%
\pgfpathlineto{\pgfqpoint{3.622484in}{0.275684in}}%
\pgfpathlineto{\pgfqpoint{3.659582in}{0.331009in}}%
\pgfpathlineto{\pgfqpoint{3.696680in}{0.398962in}}%
\pgfpathlineto{\pgfqpoint{3.733779in}{0.478449in}}%
\pgfpathlineto{\pgfqpoint{3.770877in}{0.568189in}}%
\pgfpathlineto{\pgfqpoint{3.807975in}{0.666741in}}%
\pgfpathlineto{\pgfqpoint{3.845074in}{0.772517in}}%
\pgfpathlineto{\pgfqpoint{3.882172in}{0.883817in}}%
\pgfpathlineto{\pgfqpoint{3.919270in}{0.998848in}}%
\pgfpathlineto{\pgfqpoint{3.956369in}{1.115761in}}%
\pgfusepath{stroke}%
\end{pgfscope}%
\begin{pgfscope}%
\pgfsetbuttcap%
\pgfsetmiterjoin%
\definecolor{currentfill}{rgb}{0.447059,0.470588,0.474510}%
\pgfsetfillcolor{currentfill}%
\pgfsetlinewidth{1.003750pt}%
\definecolor{currentstroke}{rgb}{0.447059,0.470588,0.474510}%
\pgfsetstrokecolor{currentstroke}%
\pgfsetdash{}{0pt}%
\pgfsys@defobject{currentmarker}{\pgfqpoint{-0.069444in}{-0.069444in}}{\pgfqpoint{0.069444in}{0.069444in}}{%
\pgfpathmoveto{\pgfqpoint{0.069444in}{-0.000000in}}%
\pgfpathlineto{\pgfqpoint{-0.069444in}{0.069444in}}%
\pgfpathlineto{\pgfqpoint{-0.069444in}{-0.069444in}}%
\pgfpathlineto{\pgfqpoint{0.069444in}{-0.000000in}}%
\pgfpathclose%
\pgfusepath{stroke,fill}%
}%
\begin{pgfscope}%
\pgfsys@transformshift{4.140005in}{1.115761in}%
\pgfsys@useobject{currentmarker}{}%
\end{pgfscope}%
\end{pgfscope}%
\begin{pgfscope}%
\pgfsetbuttcap%
\pgfsetmiterjoin%
\definecolor{currentfill}{rgb}{0.447059,0.470588,0.474510}%
\pgfsetfillcolor{currentfill}%
\pgfsetlinewidth{1.003750pt}%
\definecolor{currentstroke}{rgb}{0.447059,0.470588,0.474510}%
\pgfsetstrokecolor{currentstroke}%
\pgfsetdash{}{0pt}%
\pgfsys@defobject{currentmarker}{\pgfqpoint{-0.069444in}{-0.069444in}}{\pgfqpoint{0.069444in}{0.069444in}}{%
\pgfpathmoveto{\pgfqpoint{0.000000in}{0.069444in}}%
\pgfpathlineto{\pgfqpoint{-0.069444in}{-0.069444in}}%
\pgfpathlineto{\pgfqpoint{0.069444in}{-0.069444in}}%
\pgfpathlineto{\pgfqpoint{0.000000in}{0.069444in}}%
\pgfpathclose%
\pgfusepath{stroke,fill}%
}%
\begin{pgfscope}%
\pgfsys@transformshift{2.120003in}{2.131522in}%
\pgfsys@useobject{currentmarker}{}%
\end{pgfscope}%
\end{pgfscope}%
\begin{pgfscope}%
\pgfsetrectcap%
\pgfsetmiterjoin%
\pgfsetlinewidth{0.803000pt}%
\definecolor{currentstroke}{rgb}{0.447059,0.470588,0.474510}%
\pgfsetstrokecolor{currentstroke}%
\pgfsetdash{}{0pt}%
\pgfpathmoveto{\pgfqpoint{2.120003in}{0.100000in}}%
\pgfpathlineto{\pgfqpoint{2.120003in}{2.131522in}}%
\pgfusepath{stroke}%
\end{pgfscope}%
\begin{pgfscope}%
\pgfsetrectcap%
\pgfsetmiterjoin%
\pgfsetlinewidth{0.000000pt}%
\definecolor{currentstroke}{rgb}{0.000000,0.000000,0.000000}%
\pgfsetstrokecolor{currentstroke}%
\pgfsetstrokeopacity{0.000000}%
\pgfsetdash{}{0pt}%
\pgfpathmoveto{\pgfqpoint{4.140005in}{0.100000in}}%
\pgfpathlineto{\pgfqpoint{4.140005in}{2.131522in}}%
\pgfusepath{}%
\end{pgfscope}%
\begin{pgfscope}%
\pgfsetrectcap%
\pgfsetmiterjoin%
\pgfsetlinewidth{0.803000pt}%
\definecolor{currentstroke}{rgb}{0.447059,0.470588,0.474510}%
\pgfsetstrokecolor{currentstroke}%
\pgfsetdash{}{0pt}%
\pgfpathmoveto{\pgfqpoint{0.100000in}{1.115761in}}%
\pgfpathlineto{\pgfqpoint{4.140005in}{1.115761in}}%
\pgfusepath{stroke}%
\end{pgfscope}%
\begin{pgfscope}%
\pgfsetrectcap%
\pgfsetmiterjoin%
\pgfsetlinewidth{0.000000pt}%
\definecolor{currentstroke}{rgb}{0.000000,0.000000,0.000000}%
\pgfsetstrokecolor{currentstroke}%
\pgfsetstrokeopacity{0.000000}%
\pgfsetdash{}{0pt}%
\pgfpathmoveto{\pgfqpoint{0.100000in}{2.131522in}}%
\pgfpathlineto{\pgfqpoint{4.140005in}{2.131522in}}%
\pgfusepath{}%
\end{pgfscope}%
\begin{pgfscope}%
\definecolor{textcolor}{rgb}{0.000000,0.000000,0.000000}%
\pgfsetstrokecolor{textcolor}%
\pgfsetfillcolor{textcolor}%
\pgftext[x=4.140005in,y=0.977231in,right,]{\color{textcolor}{\rmfamily\fontsize{10.000000}{12.000000}\selectfont\catcode`\^=\active\def^{\ifmmode\sp\else\^{}\fi}\catcode`\%=\active\def%{\%}x}}%
\end{pgfscope}%
\begin{pgfscope}%
\definecolor{textcolor}{rgb}{0.000000,0.000000,0.000000}%
\pgfsetstrokecolor{textcolor}%
\pgfsetfillcolor{textcolor}%
\pgftext[x=2.207683in,y=2.131522in,left,]{\color{textcolor}{\rmfamily\fontsize{10.000000}{12.000000}\selectfont\catcode`\^=\active\def^{\ifmmode\sp\else\^{}\fi}\catcode`\%=\active\def%{\%}y}}%
\end{pgfscope}%
\begin{pgfscope}%
\definecolor{textcolor}{rgb}{0.000000,0.000000,0.000000}%
\pgfsetstrokecolor{textcolor}%
\pgfsetfillcolor{textcolor}%
\pgftext[x=2.120003in,y=2.409299in,,base]{\color{textcolor}{\rmfamily\fontsize{12.000000}{14.400000}\selectfont\catcode`\^=\active\def^{\ifmmode\sp\else\^{}\fi}\catcode`\%=\active\def%{\%}Sinus-Funktion}}%
\end{pgfscope}%
\begin{pgfscope}%
\pgfsetbuttcap%
\pgfsetmiterjoin%
\definecolor{currentfill}{rgb}{1.000000,1.000000,1.000000}%
\pgfsetfillcolor{currentfill}%
\pgfsetfillopacity{0.800000}%
\pgfsetlinewidth{1.003750pt}%
\definecolor{currentstroke}{rgb}{0.800000,0.800000,0.800000}%
\pgfsetstrokecolor{currentstroke}%
\pgfsetstrokeopacity{0.800000}%
\pgfsetdash{}{0pt}%
\pgfpathmoveto{\pgfqpoint{2.759080in}{1.812077in}}%
\pgfpathlineto{\pgfqpoint{4.042783in}{1.812077in}}%
\pgfpathquadraticcurveto{\pgfqpoint{4.070561in}{1.812077in}}{\pgfqpoint{4.070561in}{1.839855in}}%
\pgfpathlineto{\pgfqpoint{4.070561in}{2.034299in}}%
\pgfpathquadraticcurveto{\pgfqpoint{4.070561in}{2.062077in}}{\pgfqpoint{4.042783in}{2.062077in}}%
\pgfpathlineto{\pgfqpoint{2.759080in}{2.062077in}}%
\pgfpathquadraticcurveto{\pgfqpoint{2.731302in}{2.062077in}}{\pgfqpoint{2.731302in}{2.034299in}}%
\pgfpathlineto{\pgfqpoint{2.731302in}{1.839855in}}%
\pgfpathquadraticcurveto{\pgfqpoint{2.731302in}{1.812077in}}{\pgfqpoint{2.759080in}{1.812077in}}%
\pgfpathlineto{\pgfqpoint{2.759080in}{1.812077in}}%
\pgfpathclose%
\pgfusepath{stroke,fill}%
\end{pgfscope}%
\begin{pgfscope}%
\pgfsetrectcap%
\pgfsetroundjoin%
\pgfsetlinewidth{1.505625pt}%
\definecolor{currentstroke}{rgb}{0.000000,0.188235,0.368627}%
\pgfsetstrokecolor{currentstroke}%
\pgfsetdash{}{0pt}%
\pgfpathmoveto{\pgfqpoint{2.786858in}{1.950966in}}%
\pgfpathlineto{\pgfqpoint{2.925747in}{1.950966in}}%
\pgfpathlineto{\pgfqpoint{3.064636in}{1.950966in}}%
\pgfusepath{stroke}%
\end{pgfscope}%
\begin{pgfscope}%
\definecolor{textcolor}{rgb}{0.000000,0.000000,0.000000}%
\pgfsetstrokecolor{textcolor}%
\pgfsetfillcolor{textcolor}%
\pgftext[x=3.175747in,y=1.902355in,left,base]{\color{textcolor}{\rmfamily\fontsize{10.000000}{12.000000}\selectfont\catcode`\^=\active\def^{\ifmmode\sp\else\^{}\fi}\catcode`\%=\active\def%{\%}$f(x) = sin(x)$}}%
\end{pgfscope}%
\end{pgfpicture}%
\makeatother%
\endgroup%

        %\caption{PGF-Plot einer Sinus-Funktion mit \pythontex.}
        \label{fig:pythontex-pgf-plot-einer-sinus-funktion}
    \end{figure}

    Sofern ein \texttt{.pgf}-Datei verwendet wird, besteht das Problem darin, dass beim ersten Durchlauf von \emph{pdflatex} noch keine Datei existiert.
    Diese wird erst zur Laufzeit von \emph{pythontex} erstellt.
    Daher muss vorher eine Datei mit dem gleichen Namen und der Endung \texttt{.pgf} erstellt werden, damit \emph{pdflatex} diese korrekt einbinden kann.
    Alternativ kann der export über \texttt{.pdf} erfolgen.
    Durch die Einbindung mittels \texttt{\textbackslash includgraphics} wird ein Platzhalter eingebunden und kann somit kann \emph{pdflatex} ohne Fehler durchlaufen.

    \subsection{Direkte Einbindung über \pythontex}\label{subsec:direkte-einbindung-ueber-pythontex}
    Eine weitere Alternative wäre die Einbindung der Grafik direkt in der \texttt{pycode}-Umgebung, wie es in \cref{lst:pythontex-pgf-plot-direkt} dargestellt ist.
    Dieser Code erstellt den Plot direkt zur Laufzeit und bindet diesen ein (s. \cref{fig:pythontex-pgf-plot-einer-sinus-funktion-direkt-aus-pythontex}).

    \begin{lstlisting}[language=TeX, style=latexstyle, caption={direkte Einbindung von PGF-Plots mittels \pythontex}, label={lst:pythontex-pgf-plot-direkt}]
\begin{pycode}[plot2]
import os, sys
os.chdir('../src/mwe_matplotlib/pythontex')
sys.path.append(os.getcwd())

from PlotBase import plot_sinus
scalefactor = 0.9
filename = 'sinus_{}.pgf'.format(scalefactor)
plot_sinus(scalefactor*pytex.pt_to_in(pytex.context['textwidth']), filename)
pytex.add_created(filename)
print(r"\begin{figure}")
print(r"\centering")
print(r"\subimport{{}}{{{name}}}".format(name=filename))
print(r"\caption{\pythontex PGF-Plot einer Sinus-Funktion direkt aus \pythontex.}")
print(r"\label{fig:pythontex-pgf-plot-einer-sinus-funktion-direkt-aus-pythontex}")
print(r"\end{figure}")
\end{pycode}
\end{lstlisting}

\begin{pycode}[plot2]
import os, sys
os.chdir('../src/mwe_matplotlib/pythontex')
sys.path.append(os.getcwd())

from PlotBase import plot_sinus
scalefactor = 0.9
filename = 'sinus_{}.pgf'.format(scalefactor)
plot_sinus(scalefactor*pytex.pt_to_in(pytex.context['textwidth']), filename)
pytex.add_created(filename)
print(r"\begin{figure}[htbp!]")
print(r"\centering")
print(r"\subimport{{}}{{{name}}}".format(name=filename))
print(r"\caption{\pythontex PGF-Plot einer Sinus-Funktion direkt aus \pythontex.}")
print(r"\label{fig:pythontex-pgf-plot-einer-sinus-funktion-direkt-aus-pythontex}")
print(r"\end{figure}")
    \end{pycode}


    \subsection{Vergleich \pythontex und PGFPlots}\label{subsec:vergleich-von-pgf-und-pythontex}
    Der Vergleich von PGF-Plots und \pythontex-Plots ist in \cref{fig:pgf-pythontex-comparison} dargestellt.
    Dabei wird links der Plot mittels PGFPlots direkt erstellt und rechts der Plot mittels \pythontex und \matplotlib erstellt.
    Unterschiede ergeben sich vorallem in Umfang des Codes und der Art und Weise der Einbindung. Dies ist mit \matplotlib und \pythontex deutlich komplexer und aufwänder.
    Der eigentliche Umfang zur Erstellung des Plotes ist mit PGFPlots im gewählten Beispiel (\cref{lst:pgf-code}) geringer als mit \matplotlib (\cref{lst:plot-code}).
    Auch die eigentliche Einbindung in das Dokument fällt geringer aus (\cref{lst:pgf-pythontex-vergleich}).
    Weiterhin ist zu erwähnen, dass für unterschiedliche Skalierungen bei \pythontex mehrere Dateien erstellt werden müssen.

    \begin{lstlisting}[language=TeX, style=latexstyle, caption={Code zum Vergleich von PGFPlots und \pythontex}, label={lst:pgf-pythontex-vergleich}]
\begin{figure}[hb!]
    \begin{minipage}[b]{0.45\textwidth}
        \centering
        \setlength{\figurewidth}{\columnwidth}
        \subimport{../}{pgf_sinus.tex}
    \end{minipage}
    \hfill
    \begin{minipage}[b]{0.45\textwidth}
    \centering
        \begin{pycode}[plot3]
import os, sys
os.chdir('../src/mwe_matplotlib/pythontex')
sys.path.append(os.getcwd())

from PlotBase import plot_sinus
scalefactor = 1
filename = 'sinus_{}.pgf'.format(scalefactor)
plot_sinus(scalefactor*pytex.pt_to_in(pytex.context['textwidth']), filename)
pytex.add_created(filename)

print(r"\subimport{{}}{{{name}}}".format(name=filename))
        \end{pycode}
    \end{minipage}
    \caption{Vergleich von GPF direkt (links) und \pythontex (rechts).}\label{fig:pgf-pythontex-comparison}
\end{figure}
\end{lstlisting}

    \begin{figure}[hb!]
        \begin{minipage}[b]{0.45\textwidth}
            \centering
            \setlength{\figurewidth}{\columnwidth}
            \subimport{../}{pgf_sinus.tex}
        \end{minipage}
        \hfill
        \begin{minipage}[b]{0.45\textwidth}
            \centering
    \begin{pycode}[plot3]
import os, sys
os.chdir('../src/mwe_matplotlib/pythontex')
sys.path.append(os.getcwd())

from PlotBase import plot_sinus
scalefactor = 1
filename = 'sinus_{}.pgf'.format(scalefactor)
plot_sinus(scalefactor*pytex.pt_to_in(pytex.context['textwidth']), filename)
pytex.add_created(filename)

print(r"\subimport{{}}{{{name}}}".format(name=filename))
    \end{pycode}
        \end{minipage}
        \caption{Vergleich von GPF direkt (links) und \pythontex (rechts).}\label{fig:pgf-pythontex-comparison}
    \end{figure}

    \section{Fazit}
    Die Einbindung von PGF-Plots mittels \pythontex wurde in \vref{sec:vergleich-pgf-und-pythontex-plots} und \vref{sec:import-von-pgf-plots-uber-matplotlib-mittels-pythontex}  gezeigt und bietet eine gute Möglichkeit, \python-Code in \LaTeX{} einzubinden.
    Vor allem bei speziellen Grafiken, die bereits mittels \python erstellt wurden, bietet sich diese Möglichkeit an.
    Beispielsweise können so Map-Plots oder ähnliches integriert werden, welche über PGFPlots nicht oder nur sehr aufwändig erstellt werden können.

    Weiterhin kann \pythontex für die Integration von Berechnungen oder zur Datenanbindung mittels Datenbankinterfaces aus \python direkt in das \LaTeX{}-Dokument genutzt werden.
    Dies ermöglicht große Automatisierungsgrade für die Erstellung von Dokumenten. Auch die Integration des \python-Paketes \texttt{SymPy} ist ein mächtiges Werkzeug.
    Es ermöglicht die direkten Einbindung von Symbolrechnungen in \LaTeX{} und ist sogar in \pythontex mittels spezieller Befehle und Umgebungen integriert.

    Die Einbindung von aus \python heraus gespeicherten \matplotlib-Plots ist über die Berechnung der Textbreite auch möglich.
    Diese muss jedoch immer wieder neu erfolgen, sobald sich was am Layout des Dokumentes ändert.
    Aus diesem Grund ist diese Variante nur bedingt empfehlenswert.

    \Cref{sec:vergleich-pgf-und-pythontex-plots} zeigt, dass PGFPlots im Falle des gewählten Beispiels deutlich weniger Aufwand bedeuten und die Integration in \LaTeX{} geringeren Aufwand bedeutet.
    Diese Methode ist daher zu bevorzugen.

    \section{Anhang: PlotBase.py}\label{subsec:plotbase.py}

    \lstinputlisting[language=Python, style=pythonstyle, caption={Inhalt der PlotBase.py}, label={lst:PlotBase}]{PlotBase.py}


\end{document}