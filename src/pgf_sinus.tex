%! Author = Tim Häberlein
%! Organisation = Technische Universität Dresden, Professur Fahrzeugmechatronik
%! Date = 19.03.2024


% Preamble
\documentclass[class=tudscrartcl, cdfont=false]{standalone}

% Packages
\usepackage{packages}

\usepackage{pgfplots}
\usepackage{pgf}
\pgfplotsset{compat=1.9}

\newlength{\figurewidth}
%newlength{\figureheight}
\setlength{\figurewidth}{0.8\textwidth}
%\setlength{\figureheight}{0.618\figurewidth}

% Document
\begin{document}
    \begin{tikzpicture}
        \begin{axis}[
            width=\figurewidth, height=0.618\figurewidth,
            axis lines=middle,
            axis line style={-latex},
            grid=major, %both
            major grid style={cdgray},
            minor grid style={cdgrey!25},
            title=Sinus-Funktion,
            xlabel={$x$},
            ylabel={$y$},
            ymin=-1, ymax=1, minor y tick num=1,
            domain=-2*pi:2*pi,
            samples=100,
            xtick={-2*pi, -pi, 0, pi, 2*pi},
            xticklabels={$-2\pi$, $-\pi$, $0$, $\pi$, $2\pi$},
            ytick={-1, -0.5, 0, 0.5, 1},
            legend pos=north east,
            ]
            \addplot [cddarkblue, very thick]
                {sin(deg(x))};
            \addlegendentry{$f(x) = \sin(x)$}
        \end{axis}

    \end{tikzpicture}


\end{document}